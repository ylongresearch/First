%*******************************************************************************
%****************************** Second Chapter *********************************
%*******************************************************************************

\chapter{Literature Review}
The remarkable progresses on computer vision and machine learning technologies have substantially extended their applicable range. Among these applications, visual attribute learning has attracted increasingly attentions. Meanwhile, ontology engineering has been introduced to provide fundamental knowledge for learning attributes. In this chapter, the most classical machine leaning algorithms are reviewed firstly. In the second section, the state-of-the-art attribute learning methods are reviewed. Finally, existing ontology-based datasets are summarized.

 
\section{Classical Algorithms}

A typical visual task can be formalised as following: given a sample of $x$, e.g. image, video or any other information source, the goal is to find a function so that $x$ can be project to a target domain $\mathcal{Y}$. Normally, the final goal of a task has a concrete physical meaning. For instance, $x$ is an image of a human face, and $y\in\mathcal{Y}$ is a label value that indicating the identity of $x$. Such a task is known as Face Recognition. However, most of the tasks can hardly achieved by a single function: $f:x\rightarrow y$. In the next, the typical framework for such task is reviewed by three steps: feature extraction, data mining and classification.\\ 

\textbf{Base Features} \\
Due to the advance of digital technology, most of recent visual information are presented by pixels, such as images and videos. These data are known as raw data. Such data 
\textbf{Data Mining}\\

\textbf{Classification}\\

\section{Attribute Learning Methodology}

\section{Ontology-based Datasets}






